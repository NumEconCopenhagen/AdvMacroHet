%% LyX 2.4.3 created this file.  For more info, see https://www.lyx.org/.
%% Do not edit unless you really know what you are doing.
\documentclass[10pt,english,10pt,handout]{beamer}
\usepackage{lmodern}
\usepackage[T1]{fontenc}
\usepackage[utf8]{inputenc}
\setcounter{tocdepth}{1}
\usepackage{amsbsy}
\usepackage{amstext}
\usepackage{amssymb}
\usepackage[authoryear]{natbib}

\makeatletter
%%%%%%%%%%%%%%%%%%%%%%%%%%%%%% Textclass specific LaTeX commands.
% this default might be overridden by plain title style
\newcommand\makebeamertitle{\frame{\maketitle}}%
% (ERT) argument for the TOC
\AtBeginDocument{%
  \let\origtableofcontents=\tableofcontents
  \def\tableofcontents{\@ifnextchar[{\origtableofcontents}{\gobbletableofcontents}}
  \def\gobbletableofcontents#1{\origtableofcontents}
}

%%%%%%%%%%%%%%%%%%%%%%%%%%%%%% User specified LaTeX commands.



\usepackage{tikz}
\usetikzlibrary{positioning}
\usepackage{appendixnumberbeamer}

\usepackage{graphicx}
\usepackage{subfig}

\usetheme[progressbar=frametitle,block=fill,subsectionpage=progressbar]{metropolis}

% margin
\setbeamersize{text margin right=1.5cm}

% colors
%\colorlet{DarkRed}{red!70!black}
\definecolor{DarkRed}{rgb}{0.7,0,0}

\setbeamercolor{normal text}{fg=black}
\setbeamercolor{alerted text}{fg=DarkRed}
\setbeamercolor{progress bar}{fg=DarkRed}
\setbeamercolor{button}{bg=DarkRed}

% width of seperators
\makeatletter
\setlength{\metropolis@titleseparator@linewidth}{1pt}
\setlength{\metropolis@progressonsectionpage@linewidth}{1pt}
\setlength{\metropolis@progressinheadfoot@linewidth}{1pt}
\makeatother

% new alert block
\newlength\origleftmargini
\setlength\origleftmargini\leftmargini
\setbeamertemplate{itemize/enumerate body begin}{\setlength{\leftmargini}{4mm}}
\let\oldalertblock\alertblock
\let\oldendalertblock\endalertblock
\def\alertblock{\begingroup \setbeamertemplate{itemize/enumerate body begin}{\setlength{\leftmargini}{\origleftmargini}} \oldalertblock}
\def\endalertblock{\oldendalertblock \endgroup}
\setbeamertemplate{mini frame}{}
\setbeamertemplate{mini frame in current section}{}
\setbeamertemplate{mini frame in current subsection}{}
\setbeamercolor{section in head/foot}{fg=normal text.bg, bg=structure.fg}
\setbeamercolor{subsection in head/foot}{fg=normal text.bg, bg=structure.fg}

% footer
\makeatletter
\setbeamertemplate{footline}{%
    \begin{beamercolorbox}[colsep=1.5pt]{upper separation line head}
    \end{beamercolorbox}
    \begin{beamercolorbox}{section in head/foot}
      \vskip1pt\insertsectionnavigationhorizontal{\paperwidth}{}{\hskip0pt plus1filll \insertframenumber{} / \inserttotalframenumber \hskip2pt}\vskip3pt% 
    \end{beamercolorbox}%
    \begin{beamercolorbox}[colsep=1.5pt]{lower separation line head}
    \end{beamercolorbox}
}
\makeatother

% toc
\setbeamertemplate{section in toc}{\hspace*{1em}\inserttocsectionnumber.~\inserttocsection\par}
\setbeamertemplate{subsection in toc}{\hspace*{2em}\inserttocsectionnumber.\inserttocsubsectionnumber.~\inserttocsubsection\par}



% code
\usepackage{xcolor}
\usepackage{listings}

\definecolor{codegray}{rgb}{0.5,0.5,0.5}
\definecolor{background}{HTML}{F5F5F5}
\definecolor{keyword}{HTML}{4B69C6}
\definecolor{string}{HTML}{448C27}
\definecolor{comment}{HTML}{448C27}

\usepackage{inconsolata}
\lstdefinestyle{mystyle}{
    commentstyle=\color{comment},
    keywordstyle=\color{keyword},
    stringstyle=\color{string},
    basicstyle=\ttfamily,
    breakatwhitespace=false,         
    breaklines=true,                 
    captionpos=b,                    
    keepspaces=true,                                    
    numbersep=5pt,                  
    showspaces=false,                
    showstringspaces=false,
    showtabs=false,
    tabsize=4,
	showlines=true
}

\lstset{style=mystyle}
% Added by lyx2lyx
\setlength{\parskip}{\smallskipamount}
\setlength{\parindent}{0pt}

\makeatother

\usepackage{babel}
\begin{document}
\title{11. Monetary Policy in HANK\vspace{-2mm}}
\subtitle{Adv. Macro: Heterogenous Agent Models} 
\author{Jeppe Druedahl, Raphaël Huleux}
\date{2025}

{
\setbeamertemplate{footline}{} 
\begin{frame}

\maketitle

\end{frame}
}

\addtocounter{framenumber}{-1}

\section{Summing Up What We Did So Far:}

\begin{frame}{Aggregate Consumption Function in HANK}
\begin{itemize}
    \item Under some assumptions (real rate rule, real bond, no lump-sum transfers), we can derive a simpler consumption function \pause  
    \item In this case, $C_t^{hh}=C^{hh}(\{Y_s-T_s\}_{s=0}^\infty)$ depends only on net income! \pause 
    \item If $\mathbf{Y}=C^{hh}(\mathbf{Y}-\mathbf{T})+\mathbf{G}$, we thus have $d\mathbf{Y}=d\mathbf{G}+\mathbf{M}(d\mathbf{Y}-d\mathbf{T})$ \pause 
    \item The sequence space Jacobian $\mathbf{M}$ tells us everything we need to know to understand output response to a monetary policy shock!\pause 
\end{itemize}
    
\end{frame}

\begin{frame}{Plan for Today}

Monetary policy! \pause 
\begin{itemize}
   \item Derive equivalent of IKC for Monetary Policy in HANK \pause 
   \item Main results of KMV (2018) (Key paper in the literature) \pause 
   \item Study deviations from rational expectations \pause 
   \item Exercise \pause 
   \item \textbf{Dedicate last hour to the 2nd assignment.}
\end{itemize} 
\end{frame}
\section{Introduction}
\begin{frame}{Introduction}
\vspace{-2mm}
\begin{itemize}
\item <+->\textbf{Last Time: }
\begin{itemize}
\item Fiscal policy in the canonical HANK model 
\end{itemize}
\item <+->\textbf{Today: }
\begin{itemize}
\item Other pillar of stabilization policy: \textbf{Monetary policy}
\item Will use as example to study alternatives to \textbf{rational expecations}
(RE) in HANK
\end{itemize}
\item <+->\textbf{Literature: }
\begin{itemize}
\item \emph{Seminal paper}: Kaplan, Moll, Violante (2018) >>Monetary policy
according to HANK<<
\item Auclert Rognlie, Straub (2020) >>Micro jumps, macro humps<< 
\item Alves, Kaplan, Moll, Violante (2020) >>A further look at the propagation
of monetary policy shocks in HANK<<
\end{itemize}
\end{itemize}
\end{frame}
%

\section{Monetary Policy in HANK}

\begin{frame}{Monetary Policy}
\begin{itemize}
\item <+->Introducing heterogeneous agents into the standard NK model \textbf{fundamentally}
changes the transmission of Fiscal Policy
\begin{itemize}
\item \emph{Potentially} more effective 
\item Important whether policy is deficit financed or tax financed 
\end{itemize}
\item <+->What about monetary policy? 
\end{itemize}
\end{frame}
%
\begin{frame}{Model}
\begin{itemize}
\item <+->\textbf{Last time}: Canonical HANK model 
\item <+->Very close to standard \textbf{NK} except for:
\begin{itemize}
\item HA instead of RA
\item Sticky wages 
\item Government
\end{itemize}
\item <+->Today: Monetary policy - don't really need a government?
\begin{itemize}
\item <+->Issue: If we remove government no liquidity for households to
save in
\begin{itemize}
\item Fine in RA, issue in HA with borrowing constraint at $a=0$
\end{itemize}
\item <+->\textbf{Solution: }Firm equity 
\end{itemize}
\end{itemize}
\end{frame}
%
\begin{frame}{Households}
\begin{itemize}
\item \textbf{Household problem:}
\begin{align*}
v_{t}(z_{t},a_{t-1}) & =\max_{c_{t}}\frac{c_{t}^{1-\sigma}}{1-\sigma}-\varphi\frac{\ell_{t}^{1+\nu}}{1+\nu}+\beta\mathbb{E}_{t}\left[v_{t+1}(z_{t+1},a_{t})\right]\\
\text{s.t. }a_{t}+c_{t} & =(1+r_{t}^{a})a_{t-1}+Z_{t}z_{t}+\chi_{t}\\
\log z_{t+1} & =\rho_{z}\log z_{t}+\psi_{t+1}\,\,\,,\psi_{t}\sim\mathcal{N}(\mu_{\psi},\sigma_{\psi}),\,\mathbb{E}[z_{t}]=1\\
a_{t} & \geq0
\end{align*}
\item with $Z_{t}=w_{t}\ell_{t}$ - real labor income
\item \textbf{decisions:} Consumption-saving, $c_{t}$ (and $a_{t}$)
\item \textbf{Union decision:} Labor supply, $\ell_{t}$
\item \textbf{Aggregate Consumption: $C_{t}^{hh}=\int c_{t}d\mathcal{D}_{t}$}
\item \textbf{Consumption function: $C_{t}^{hh}=C^{hh}\left(\{r_{s}^{a},Z_{s},\chi_{s}\}_{s=0}^{\infty}\right)$}
\end{itemize}
\end{frame}
%
\begin{frame}{Firms}
\begin{itemize}
\item \textbf{Production and profits:}
\begin{align*}
Y_{t} & =L_{t}\\
\Pi_{t} & =Y_{t}-w_{t}L_{t}
\end{align*}
\item Optimize subject to demand curve (monopolistic competition) 
\item \textbf{First order condition:}
\[
w_{t}=\frac{1}{\mu}
\]
\item where $\mu>1=$ markup - firms make positive profits in equilibrium
\end{itemize}
\end{frame}
%
\begin{frame}{Mutual fund I}

\begin{itemize}
\item <+->Mutual fund collect households savings $A_{t}$ and invest in
firm equity
\item <+->Firm $j$ has ownership shares $\upsilon_{j,t}$ with price $p_{j,t}^{D}$
\item <+->If you own shares in the firm you get profits/dividends $\Pi_{j,t}$
\item <+->Shares sum to 1, $\int\upsilon_{j,t}dj=1$
\item <+->Firms are gonna be symmetric in eq., $p_{j,t}^{D}=p_{t}^{D}$
\item <+->Total value of firm equity is then $\int p_{t}^{D}\upsilon_{j,t}dj=p_{t}^{D}$
\end{itemize}
\end{frame}
%
\begin{frame}{Mutual fund II}
\begin{itemize}
\item <+->\textbf{Problem}:
\[
\max_{\upsilon_{j,t}}\int\left(\Pi_{j,t+1}+p_{j,t+1}^{D}\right)\upsilon_{j,t}-\left(1+r_{t+1}^{a}\right)A_{t}
\]
\item <+->\textbf{Subject} to balance sheet:
\[
\int p_{j,t}^{D}\upsilon_{j,t}dj=A_{t}
\]
\item <+->\textbf{FOC}:
\[
p_{t}^{D}=\frac{\Pi_{t+1}+p_{t+1}^{D}}{1+r_{t}}
\]
\item <+->where $r_{t}=E_{t}r_{t+1}^{a}$ the ex-ante interest rate
\item <+->Price of equity can be written as (assume $r_{t}=r$): $p_{t}^{D}=\sum_{s=0}^{\infty}\left(1+r\right)^{-s}\Pi_{t+s}$
\begin{itemize}
\item {\footnotesize Asset price today reflect discounted sum of future profits}{\footnotesize\par}
\end{itemize}
\item <+->{\footnotesize Valuation effects: As with nominal gov bonds:}\\
{\footnotesize\vspace{-0.4cm}}
\[
1+r_{t}^{a}=\begin{cases}
\frac{\Pi_{0}+p_{0}^{D}}{p_{ss}^{D}} & t=0\\
1+r_{t-1} & t>0
\end{cases}
\]
\end{itemize}
\end{frame}
%
\begin{frame}{Union}
\begin{itemize}
\item Everybody works the same:
\begin{align*}
\ell_{t} & =L_{t}^{hh}
\end{align*}
\item Maximization subject to wage adjustment cost imply a \textbf{New Keynesian
Wage (Phillips) Curve} (NKWPC or NKWC)
\[
\pi_{t}^{w}=\kappa\left(\varphi\left(L_{t}^{hh}\right)^{\nu}-\frac{1}{\mu}\left(1-\tau_{t}\right)w_{t}\left(C_{t}^{hh}\right)^{\text{-\ensuremath{\sigma}}}\right)+\beta\pi_{t+1}^{w}
\]
\end{itemize}
\end{frame}
%
\begin{frame}{Central bank}
\begin{itemize}
\item <+->Two options for monetary policy 
\item <+->\textbf{1. }Government bonds are nominel, CB chooses nominel
interest rate: 
\[
i_{t}=i_{ss}+\phi\pi_{t}
\]
\item <+->And fisher equation links nominal rate $i$ to real rate $r$:
\[
1+r_{t}=\frac{1+i_{t}}{1+\pi_{t+1}}
\]
\item <+->\textbf{2. Alternative:} Real rate rule. CB chooses real rate
$r_{t}$ directly 

\[
r_{t}=r_{ss}+\left(\phi-1\right)\pi_{t}
\]

\end{itemize}
\end{frame}
%
\begin{frame}{Market clearing}
\begin{enumerate}
\item Asset market: $p_{t}^{D}=A_{t}^{hh}$
\item Labor market: $L_{t}=L_{t}^{hh}$
\item Goods market: $Y_{t}=C_{t}^{hh}$
\end{enumerate}
\end{frame}
%
\begin{frame}{The consumption function}

\begin{itemize}
\item <+->Model features a consumption function:
\[
C_{t}^{hh}=C_{t}^{hh}\left(\left\{ r_{s}^{a},Z_{s}\right\} _{s=0}^{\infty}\right)\Rightarrow\boldsymbol{C}^{hh}=C^{hh}\left(\boldsymbol{r}^{a},\boldsymbol{Z}\right)
\]
\item <+->Linearize around steady state:
\[
d\boldsymbol{C}=\boldsymbol{M}d\boldsymbol{Z}+\boldsymbol{M}_{r^{a}}d\boldsymbol{r}^{a}
\]
\item <+->As discussed in last lecture, can split overall effect of asset
returns $d\boldsymbol{r}^{a}$ into intertemporal substitution effect
(ex-ante $r$) and a capital gain effect at time 0:
\[
d\boldsymbol{C}=\boldsymbol{M}d\boldsymbol{Z}+\boldsymbol{M}_{r}d\boldsymbol{r}+\boldsymbol{m}dcap_{0}
\]
\item <+->Note: $\boldsymbol{m}$ is a vector not matrix (multiplies onto
scalar $dcap_{0}$, not vector)
\end{itemize}
\end{frame}
%
\begin{frame}{Interest rate Jacobians}

 \begin{minipage}{0.48\textwidth}         
\centering         
\includegraphics[width=\textwidth]{figs/M_r.pdf}         
\end{minipage}    
\hfill     
\begin{minipage}{0.48\textwidth}       
 \centering      
  \includegraphics[width=\textwidth]{figs/m_cap.pdf}    
 \end{minipage}
\end{frame}
%
\begin{frame}{Monetary policy in sequence-space}
\begin{itemize}
\item <+->Write real labor income as $Z_{t}=w_{t}L_{t}=\frac{1}{\mu}Y_{t}\Rightarrow d\boldsymbol{Z}=\frac{1}{\mu}d\boldsymbol{Y}$
\item <+->Linearize goods market clearing:
\[
d\boldsymbol{Y}=\boldsymbol{M}_{r}d\boldsymbol{r}+\frac{1}{\mu}\boldsymbol{M}d\boldsymbol{Y}+\boldsymbol{m}dcap_{0}
\]
\item <+->For small capital gains, solution is:
\[
d\boldsymbol{Y}=\left(\boldsymbol{I}-\frac{1}{\mu}\boldsymbol{M}\right)^{-1}\boldsymbol{M}_{r}d\boldsymbol{r}\equiv\mathcal{M}\boldsymbol{M}_{r}d\boldsymbol{r}
\]
\item <+->Note: \textbf{can} multiplier invert this PV of columns in $\frac{1}{\mu}\boldsymbol{M}$
is not 1 when $\mu>1$!)
\item <+->Monetary policy operates through: 
\begin{itemize}
\item \textbf{Direct} (partial eq., $\boldsymbol{M}_{r}$) effect 
\item \textbf{Indirect} (general eq., $\boldsymbol{M}$) effect 
\end{itemize}
\item <+->\textbf{Q1}: Sign? Positive/negative? 
\item <+->\textbf{Q2}: Do you expect the effects of monetary policy on
output to be larger in HANK than RANK?
\end{itemize}
\end{frame}
%
\begin{frame}{HANK-RANK equivalence}
\begin{itemize}
\item <+->Assume logarithmic utility $u\left(c\right)=\log\left(c\right)$ 
\item <+->\textbf{Proposition}: Above model features \textbf{exact} equivalence
between RANK and HANK for the response of aggregates w.r.t a monetary
policy shock (\emph{Werning 2015}) 
\item <+->\textbf{... but transmission channel is different} 
\item <+->Decompose $d\boldsymbol{Y}$ into direct and indrect effect using
$d\boldsymbol{Y}^{j}=\boldsymbol{M}_{r^{a}}^{j}d\boldsymbol{r^{a}}+\boldsymbol{M}^{j}d\boldsymbol{Z}$
for $j\in\left\{ HA,RA\right\} $\\
\begin{figure}[H]     
\centering      
\includegraphics[width=0.9\linewidth]{figs/HANK_Monpol_decomp.pdf}      
\end{figure}
\end{itemize}
\end{frame}
%
\begin{frame}{HANK-RANK equivalence}
\begin{itemize}
\item <+->In basic HANK model:
\begin{itemize}
\item Monetary policy has same effectiveness as in RANK 
\item But transmission different: Indirect income effects more important
than in RANK
\end{itemize}
\item <+->Exact \textbf{equivalence }is the product of a number of simplifying
assumptions:
\begin{itemize}
\item Linear production function 
\item No investment 
\item Log utility 
\item Equal incidence of labor income 
\item No government debt
\end{itemize}
\item <+->How does the effectiveness of monetary policy look in more realistic
models?
\begin{itemize}
\item Kaplan, Moll, Violante (2018) >>Monetary policy according to HANK<<
\item Auclert Rognlie, Straub (2020) >>Micro jumps, macro humps<< 
\end{itemize}
\end{itemize}
\end{frame}
%

\section{KMV 2018}
\begin{frame}{Monetary policy according to HANK}
\begin{itemize}
\item <+->Kaplan, Moll, Violante (2018) is a seminal paper in the HANK
litterature
\begin{itemize}
\item The term HANK originates from this paper
\end{itemize}
\item <+->They study the transmission of monetary policy in medium scale
HANK model 
\item <+->Follows Kaplan \& Violante (2014) closely
\begin{itemize}
\item See lecture 2 
\item Household can hold both liquid and illiquid assets 
\item Model features both \textbf{poor} and \textbf{wealthy} Hand-to-mouth
households
\end{itemize}
\end{itemize}
\end{frame}
%
\begin{frame}{Household problem}
\begin{itemize}
\item <+->Households solve (here converted to discrete time, paper in cont.
time):
\begin{align*}
V_{t}\left(a_{t-1},b_{t-1},z_{t}\right) & =\max_{c_{t},a_{t},b_{t}}u\left(c_{t},\ell_{t}\right)+\beta E_{t}V_{t+1}\left(a_{t},b_{t},z_{t+1}\right)\\
b_{t} & +c_{t}=(1-\tau_{t})w_{t}z_{t}\ell_{t}+\left(1+r_{t}^{b}\right)b_{t-1}-d_{t}-\chi(d_{t},a_{t-1})\\
a_{t} & =\left(1+r_{t}^{a}\right)a_{t-1}+d_{t}\\
b_{t}\geq-\bar{b} & \quad a_{t}\geq0.\\
\end{align*}
\item <+->with $b_{t}$=liquid asset, $a_{t}$=illiquid assets, $d_{t}$=deposits
into illiquid asset, $\chi(d_t,a_{t-1})$ a convex cost
\item <+->Return on illiquid asset $r_{t}^{a}$ Return on liquid asset
$r_{t}^{b}$
\begin{itemize}
\item Household will prefer to hold $a_{t}$ due to superior return
\item But not good for consumption smoothing as they have to pay adjustment
cost to use $a_{t}$ for smoothing against shocks
\item Some HHs will be wealthy hand-to-mouth
\end{itemize}
\end{itemize}
\end{frame}
%
\begin{frame}{MPCs}
\begin{itemize}
\item 1) MPCs for different sizes of stimulus checks, 2) MPCs across the
wealth distribution\\
\begin{figure}[H]     
\centering      
\includegraphics[width=0.9\linewidth]{figs/KMVMPCs.png}      
\end{figure}
\end{itemize}
\end{frame}
%
\begin{frame}{Direct vs indirect effects}
\begin{itemize}
\item <+->Amplification in HANK (elasticity of $C^{HANK}$ $=-2.9$ vs
$C^{RANK}$ $=-2.07$) 
\item <+->Baseline HANK: Indirect effects account for majority of transmission
($\approx80\%$)\\
\begin{figure}[H]     
\centering      
\includegraphics[width=0.8\linewidth]{figs/KMVTable7.png}      
\end{figure}
\end{itemize}
\end{frame}
%

\section{Expectations}
\begin{frame}{Micro Jumps, Macro Humps}
\begin{itemize}
\item <+->Auclert, Rognlie and Straub (2020) >>Micro jumps, macro humps<< 
\item <+->Estimate parameters in quantitative HANK model to match estimated
effects of causal monetary policy shock
\item <+->Main hurdle: Empirical response of $C,Y$ is hump-shaped to monetary
policy shock.
\item <+->Want a model that simultaneously match hump-shaped agg. response
to $r$ and iMPC moment\\
\begin{figure}[H]     
\centering      
\includegraphics[width=0.7\linewidth]{figs/MJMCFig1.png}      
\end{figure}
\end{itemize}
\end{frame}
%
\begin{frame}{The problem}
\begin{itemize}
\item Standard model does \textbf{not }give\textbf{ }hump shaped\textbf{
}for $C$ to standard shock 
\item Does not matter if \textbf{shock} is hump shaped or not\\
\begin{figure}[H]     
\centering      
\includegraphics[width=0.9\linewidth]{figs/HANK_Monpol_shape.pdf}      
\end{figure}
\end{itemize}
\end{frame}
%
\begin{frame}{The solution: RANK}
\begin{itemize}
\item <+->Solution in RANK litterature: Habits in utility function:
\begin{align*}
\sum_{t=0}^{\infty} & \beta^{t}u\left(C_{t}-\gamma C_{t-1}\right)\\
\Rightarrow u'\left(C_{t}-\gamma C_{t-1}\right)= & \beta R_{t+1}u'\left(C_{t+1}-\gamma C_{t}\right)
\end{align*}
\item <+->Generates persistence in C response to shocks because household
don't want to deviate too much from last periods consumption level 
\item <+->\textbf{However}: Does not work in HANK because it kills iMPCs:\\
\begin{figure}[H]     
\centering      
\includegraphics[width=0.5\linewidth]{figs/MJMCFig2.png}      
\end{figure}
\end{itemize}
\end{frame}
%
\begin{frame}{Deviations from alternative expecations}
\begin{itemize}
\item <+->Solution: \textbf{Deviate} from \textbf{rational expecations}
(RE)
\item <+->Assume households have imperfect expecations about \textbf{changes}
in \textbf{aggregate variables} ($Z,r$)
\begin{itemize}
\item Implies that steady state is unaffected 
\item Still rational expecations w.r.t idiosyncratic income shocks 
\end{itemize}
\item <+->Will only implement this to first-order (e.g. linear approximations)
\begin{itemize}
\item Much more difficult if we want full non-linear solution
\end{itemize}
\end{itemize}
\end{frame}
%
\begin{frame}{Income Jacobian}
\begin{itemize}
\item <+->Example: Response of aggregate consumption $\boldsymbol{C}$
to change in agg. income $\boldsymbol{Z}$
\[
d\boldsymbol{C}=\boldsymbol{M}d\boldsymbol{Z}
\]
\item <+->where $\boldsymbol{M}$ is jacobian with rational expectations:
\[
\boldsymbol{M}=\begin{bmatrix}\frac{\partial C_{0}}{\partial Z_{0}} & \frac{\partial C_{0}}{\partial Z_{1}} & \frac{\partial C_{0}}{\partial Z_{2}} & \cdots\\
\frac{\partial C_{1}}{\partial Z_{0}} & \frac{\partial C_{1}}{\partial Z_{1}} & \frac{\partial C_{1}}{\partial Z_{2}} & \cdots\\
\frac{\partial C_{2}}{\partial Z_{0}} & \frac{\partial C_{2}}{\partial Z_{1}} & \frac{\partial C_{2}}{\partial Z_{2}} & \cdots\\
\vdots & \vdots & \vdots & \ddots
\end{bmatrix}
\]
\item <+->Note that elements above diagonal are affected by expectations
(i.e. they concern the \textbf{future})
\begin{itemize}
\item Elements on and below diagonal reflect changes in income \textbf{today}
or in the \textbf{past} (known by HHs)
\end{itemize}
\end{itemize}
\end{frame}
%
\begin{frame}{Expectations matrix}
\begin{itemize}
\item <+->Introduce expectations matrix $\boldsymbol{E}$:
\[
\boldsymbol{E}=\begin{bmatrix}1 & * & * & * & \cdots\\
1 & 1 & * & * & \cdots\\
1 & 1 & 1 & * & \cdots\\
1 & 1 & 1 & 1 & \cdots\\
\vdots & \vdots & \vdots & \vdots & \ddots
\end{bmatrix}
\]
\item <+->Element $t,s$ ($E_{t,s})$ captures average date-$t$ exp. about
shock to $Z$ at date $s$. 
\begin{itemize}
\item $E_{t,s}dZ_{s}$ is then the expected value of $dZ_{s}$ at date $t$
\item First column: Exp. of HHs at all dates w.r.t $dZ_{0}$
\item Second column: Exp. of HHs at all dates w.r.t $dZ_{1}$...
\end{itemize}
\item <+->How to get jacobian $\hat{\boldsymbol{M}}$ associated with $\boldsymbol{E}$?
\end{itemize}
\end{frame}
%
\begin{frame}{Stylized Example I}
\begin{itemize}
\item <+->Expectations matrix:
\[
\boldsymbol{E}=\begin{bmatrix}1 & 0.4 & 0.3 & \cdots\\
1 & 1 & 0.6 & \cdots\\
1 & 1 & 1 & \cdots\\
\vdots & \vdots & \vdots & \ddots
\end{bmatrix}
\]
\item <+->At $t=0$ expected path of $d\boldsymbol{Z}$ is $\left\{ 1\cdot dZ_{0},0.4\cdot dZ_{1},0.3\cdot dZ_{2}\right\} $
\item <+->At $t=1$ expected path of $d\boldsymbol{Z}$ is $\left\{ 1\cdot dZ_{0},1\cdot dZ_{1},0.6\cdot dZ_{2}\right\} $
\item <+->What is response of $\boldsymbol{C}$? 
\item <+->Period 0 with RE:
\[
dC_{0}=\frac{\partial C_{0}}{\partial Z_{0}}dZ_{0}+\frac{\partial C_{0}}{\partial Z_{1}}dZ_{1}+\frac{\partial C_{0}}{\partial Z_{2}}dZ_{2}+\ldots
\]
\item <+->With alternative $\boldsymbol{E}$:
\[
d\hat{C}_{0}=\frac{\partial C_{0}}{\partial Z_{0}}dZ_{0}+0.4\cdot\frac{\partial C_{0}}{\partial Z_{1}}dZ_{1}+0.3\cdot\frac{\partial C_{0}}{\partial Z_{2}}dZ_{2}+\ldots
\]
\end{itemize}
\end{frame}
%
\begin{frame}{Stylized Example II}
\begin{itemize}
\item <+->$d\hat{C}_{0}$ simple to get. What about $d\hat{C}_{1}?$ 
\item <+->With RE we have:
\[
dC_{1}=\frac{\partial C_{1}}{\partial Z_{0}}dZ_{0}+\frac{\partial C_{1}}{\partial Z_{1}}dZ_{1}+\frac{\partial C_{1}}{\partial Z_{2}}dZ_{2}+\ldots
\]
\item <+->With Alternative $\boldsymbol{E}$:{\footnotesize
\[
d\hat{C}_{1}=\underbrace{\frac{\partial C_{1}}{\partial Z_{0}}dZ_{0}}_{\text{Past shock}}+\underbrace{0.4\frac{\partial C_{1}}{\partial Z_{1}}dZ_{1}+\left(1-0.4\right)\frac{\partial C_{0}}{\partial Z_{0}}dZ_{1}}_{\text{Shock "today"}}+\underbrace{0.3\frac{\partial C_{1}}{\partial Z_{2}}dZ_{2}+\left(0.6-0.3\right)\frac{\partial C_{0}}{\partial Z_{1}}dZ_{2}}_{\text{Future shock}}
\]
}{\footnotesize\par}
\item <+->Intuition:
\begin{itemize}
\item <+->Past shock: Fully known, so standard effect
\item <+->Present shock: Weighted average of forward looking RE part and
>>myopic<< surprise
\item <+->Future shock: Initial RE part from period 0 (weight: $0.3$)
and revision of expectations (weight: $0.6-0.3$)
\end{itemize}
\end{itemize}
\end{frame}
%
\begin{frame}{General formula}
\begin{itemize}
\item <+->At first glance seems hard to implement
\item <+->... but we have a general formula to get $\hat{\boldsymbol{M}}$
given $\boldsymbol{E}$
\item <+->$\hat{\boldsymbol{M}}$ matrix with expectations matrix $\boldsymbol{E}:$
\[
\hat{M}_{t,s}=\sum_{\tau=0}^{\min\left\{ t,s\right\} }\underbrace{\left(E_{\tau,s}-E_{\tau-1,s}\right)M_{t-\tau,s-\tau}}_{\text{date-\ensuremath{t} effect of date-\ensuremath{\tau} expectation revision of date-\ensuremath{s} shock}}
\]
\item <+->with $E_{-1,s}=0$ by convention
\item <+->\textbf{Fast} and \textbf{easy} to implement
\end{itemize}
\end{frame}
%
\begin{frame}{Examples}
\begin{itemize}
\item <+->Examples:{\footnotesize
\[
\boldsymbol{E}^{\text{RE}}=\begin{bmatrix}1 & 1 & 1 & 1 & \cdots\\
1 & 1 & 1 & 1 & \cdots\\
1 & 1 & 1 & 1 & \cdots\\
1 & 1 & 1 & 1 & \cdots\\
\vdots & \vdots & \vdots & \vdots & \ddots
\end{bmatrix},\quad\quad\quad\boldsymbol{E}^{\text{Myopic}}=\begin{bmatrix}1 & 0 & 0 & 0 & \cdots\\
1 & 1 & 0 & 0 & \cdots\\
1 & 1 & 1 & 0 & \cdots\\
1 & 1 & 1 & 1 & \cdots\\
\vdots & \vdots & \vdots & \vdots & \ddots
\end{bmatrix}
\]
}{\footnotesize\par}
\item <+->Two extremes: 
\begin{itemize}
\item {\small\textbf{Rational}}{\small{} expectations: Households are fully
informed about the future path of }{\small\textbf{$Z$ }}{\small from
the moment the shock manifests}{\small\par}
\item {\small\textbf{Myopic}}{\small{} expectations: Households are not forward
looking w.r.t aggregates. Every change in $Z$ is a surprise.}{\small\par}
\end{itemize}
\item <+->Implied jacobians:{\footnotesize
\[
\boldsymbol{\hat{M}}^{\text{RE}}=\boldsymbol{M}=\begin{bmatrix}\frac{\partial C_{0}}{\partial Z_{0}} & \frac{\partial C_{0}}{\partial Z_{1}} & \frac{\partial C_{0}}{\partial Z_{2}} & \cdots\\
\frac{\partial C_{1}}{\partial Z_{0}} & \frac{\partial C_{1}}{\partial Z_{1}} & \frac{\partial C_{1}}{\partial Z_{2}} & \cdots\\
\frac{\partial C_{2}}{\partial Z_{0}} & \frac{\partial C_{2}}{\partial Z_{1}} & \frac{\partial C_{2}}{\partial Z_{2}} & \cdots\\
\vdots & \vdots & \vdots & \ddots
\end{bmatrix},\quad\quad\quad\boldsymbol{\hat{M}}^{\text{Myopic}}=\begin{bmatrix}\frac{\partial C_{0}}{\partial Z_{0}} & 0 & 0 & \cdots\\
\frac{\partial C_{1}}{\partial Z_{0}} & \frac{\partial C_{0}}{\partial Z_{0}} & 0 & \cdots\\
\frac{\partial C_{2}}{\partial Z_{0}} & \frac{\partial C_{1}}{\partial Z_{0}} & \frac{\partial C_{0}}{\partial Z_{0}} & \cdots\\
\vdots & \vdots & \vdots & \ddots
\end{bmatrix}
\]
}{\footnotesize\par}
\end{itemize}
\end{frame}
%
\begin{frame}{Jacobians}
\begin{itemize}
\item Jacobian of C w.r.t Z \vspace{1cm}\\
 \begin{minipage}{0.45\textwidth}         
\centering         
\includegraphics[width=\textwidth]{figs/M_RE.pdf}         
\end{minipage}    
\hfill     
\begin{minipage}{0.45\textwidth}       
 \centering      
  \includegraphics[width=\textwidth]{figs/M_myopic.pdf}    
 \end{minipage}
\end{itemize}
\end{frame}
%
\begin{frame}{Solving GE with non-RE expecations}
\begin{itemize}
\item <+->Given some exp. matrix $\boldsymbol{E}$ we can construct alternative
jacobians $\boldsymbol{\hat{M}},\boldsymbol{\hat{M}}_{r}$ 
\item <+->Solve for GE using these jacobians instead of RE jacobians $\boldsymbol{M},\boldsymbol{M}_{r}$
($\boldsymbol{X}$=shock):
\[
\boldsymbol{H}\left(\boldsymbol{U},\boldsymbol{X}\right)=0\Rightarrow d\boldsymbol{U}=-\boldsymbol{\hat{H}}_{U}^{-1}\hat{\boldsymbol{H}}_{X}d\boldsymbol{X}
\]
\item <+->In our example:
\begin{align*}
\boldsymbol{Y} & -\boldsymbol{C}\left(\boldsymbol{r},\frac{1}{\mu}\boldsymbol{Y}\right)=0\\
\Rightarrow & d\boldsymbol{Y}=\left(\boldsymbol{I}-\frac{1}{\mu}\boldsymbol{\hat{M}}\right)^{-1}\boldsymbol{\hat{M}}_{r}d\boldsymbol{r}
\end{align*}
\item <+->where $-\boldsymbol{\hat{H}}_{U}^{-1}=\left(\boldsymbol{I}-\frac{1}{\mu}\boldsymbol{\hat{M}}\right)^{-1}$
and $\hat{\boldsymbol{H}}_{X}=-\boldsymbol{\hat{M}}_{r}$
\end{itemize}
\end{frame}
%
\begin{frame}{Non-RE expecations in GEModelTools}
\begin{itemize}
\item <+->How to implement in GEModelTools?
\begin{itemize}
\item Currently no built in way to handle 
\end{itemize}
\item <+->Work around (see exercise):
\begin{enumerate}
\item <+->Compute all Jacobians for household block 
\begin{itemize}
\item \lstinline{model._compute_jac_hh()}
\item If using RA/TA instead of HA must manually compute jac
\end{itemize}
\item <+->Construct Expectation matrix $\boldsymbol{E}$ and compute $\boldsymbol{\hat{M}}$
by modyfing RE jacobians in \lstinline{model.jac_hh} 
\item <+->Overwrite jacobians \lstinline{model.jac_hh} with $\boldsymbol{\hat{M}}$
for each output/input to household block
\item <+->Compute all Jacobians w.r.t unknowns and shocks, but \textbf{not}
for household block 
\begin{itemize}
\item \lstinline{model.compute_jacs(skip_hh=True,skip_shocks=False)}
\item GEModelTools will automatically use whatever jacobian is in \lstinline{model.jac_hh}
to construct Jacobians $\boldsymbol{H}_{U},\boldsymbol{H}_{Z}$
\end{itemize}
\item <+->Solve for IRFs:
\begin{itemize}
\item \lstinline{model.find_IRFs(shocks=[x])}
\end{itemize}
\end{enumerate}
\end{itemize}
\end{frame}
%
\begin{frame}{Back to Auclert, Rognlie, Straub (2020) - Sticky expectations}
\begin{itemize}
\item <+->Auclert, Rognlie, Straub (2020) use \textbf{sticky information/expectations}
(Mankiw and Reis (2002)) 
\item <+->Only a fraction $1-\theta$ of HHs update their information set
about the aggregate economy each period 
\begin{itemize}
\item Only learn full path of shock $d\boldsymbol{r},d\boldsymbol{Z}$ if
you update 
\item 1. period: $1-\theta$ update and learn full path 
\item 2. period: $\theta\left(1-\theta\right)$ update, so $1-\theta+\theta\left(1-\theta\right)=1-\theta^{2}$
have full info
\end{itemize}
\item <+->Expectations matrix:
\[
\boldsymbol{E}=\begin{pmatrix}1 & 1-\theta & 1-\theta & \dots\\
1 & 1 & 1-\theta^{2} & \dots\\
1 & 1 & 1 & \dots\\
\vdots & \vdots & \vdots & \ddots
\end{pmatrix}
\]
\end{itemize}
\end{frame}
%
\begin{frame}{Sticky expectations}
\begin{itemize}
\item <+->Properties:
\begin{itemize}
\item Response of consumption at $0$ to $Z_{1}$ is $\left(1-\theta\right)\frac{\partial C_{0}}{\partial Z_{1}}$
\item Response of consumption at $1$ to $Z_{1}$is $\left(1-\theta\right)\frac{\partial C_{1}}{\partial Z_{1}}+\theta\frac{\partial C_{0}}{\partial Z_{0}}$
and so forth
\end{itemize}
\item <+->$\theta=0$ gives us RE, $\theta=1$ gives us myopic behavior. 
\item <+->Since households perfectly observe income changes \textbf{today
and in past }iMPCs are preserved
\begin{itemize}
\item \textbf{Unlike} habit formation 
\end{itemize}
\end{itemize}
\end{frame}
%
\begin{frame}{Estimation}
\begin{itemize}
\item <+->Auclert, Rognlie, Straub (2020) fomulate full HANK model with:
\begin{itemize}
\item Investment 
\item Sticky wages + prices
\item Government
\end{itemize}
\item <+->Estimate parameters to match empirical evidence on causally identified
monetary policy shock in the US (Romer \& Romer shock) \\
\begin{figure}[H]     
\centering      
\includegraphics[width=0.7\linewidth]{figs/MJMCFig3.png}      
\end{figure}
\item <+->Estimate $\theta=0.935$ $\Rightarrow$ \textbf{Large }deviation
from RE
\end{itemize}
\end{frame}
%
\begin{frame}{RE vs. Non-RE}
\begin{itemize}
\item Why we need sticky expectations in order to match empirical response\\
\begin{figure}[H]     
\centering      
\includegraphics[width=0.85\linewidth]{figs/MJMCFig4.png}      
\end{figure}
\end{itemize}
\end{frame}
%
\begin{frame}{Direct and indirect effects}
\begin{itemize}
\item Can decompose $C$ into direct and indirect as before\\
\begin{figure}[H]     
\centering      
\includegraphics[width=0.85\linewidth]{figs/MJMCFig6.png}      
\end{figure}
\item In the estimated model with sticky expectations indirect effect is
by far the most important driver of consumption
\end{itemize}
\end{frame}
%
\begin{frame}{Importance of Investment}
\begin{itemize}
\item Importance of indirect effects in HANK partly comes from \emph{investment}\\
\emph{\begin{figure}[H]     
\centering      
\includegraphics[width=0.9\linewidth]{figs/MJMCFig5.png}      
\end{figure}}
\end{itemize}
\end{frame}

\begin{frame}{Summing-Up}
\begin{enumerate}
    \item We can use our SSJ machinery to include deviations from RE \pause
    \item To do so, we just need define an expectation matrix, and recompute the Jacobian accordingly \pause 
    \item This allows HANK models to match the observed "hump-shape" in the data 
\end{enumerate}    
\end{frame}

\section{Exercise}
\begin{frame}{Exercise}

Consider the HANK model described in section 2
\begin{enumerate}
\item {\small Compare a monetary policy shock in HANK and RANK. Decompose
the response in HANK into direct and indirect effects using the household
Jacobians}{\small\par}
\item {\small Solve for a monetary policy shock in HANK and RANK with myopic
expectations w.r.t $\boldsymbol{r},\boldsymbol{Z},$only $\boldsymbol{r}$
and only $\boldsymbol{Z}$}{\small\par}
\item {\small Solve for a monetary policy shock in HANK and RANK with sticky
expectations w.r.t $\boldsymbol{r},\boldsymbol{Z},$only $\boldsymbol{r}$
and only $\boldsymbol{Z}$}{\small\par}
\item {\small Consider a model where household hold nominal government debt
instead. Relax the borrowing constraint to -1, $\underline{a}=-1$
and solve for a monetary policy shock (assume rational expectations).
Does the presence of household debt amplify or dampen the effects
of monetary policy?}{\small\par}
\end{enumerate}
\end{frame}


%

\section{Summary}
\begin{frame}{Summary and next week}
\begin{itemize}
\item \textbf{Today: }
\begin{itemize}
\item Monetary policy in HANK 
\item Alternatives to rational expecations, and how to implement them using
jacobians 
\end{itemize}
\item \textbf{Next week: }HANK + unemployment risk in GE (\textbf{JD})
\item \textbf{Homework:}
\begin{enumerate}
\item Work on exercise
\end{enumerate}
\end{itemize}
\end{frame}
%

\end{document}
